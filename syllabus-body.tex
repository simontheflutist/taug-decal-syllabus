%\begin{description}
%	\item[Faciltators] Simon Kuang (\url{simontheflutist@berkeley.edu})\\
%	Simon Kuang (\url{simontheflutist@berkeley.edu})\\
%\end{description}
%\noindent
\begin{tabular}{>{\bfseries}ll}
	 Facilitators & Simon Kuang (\url{simontheflutist@berkeley.edu}) \\
		& Jacquelyn Vasantachat (\url{jvasantachat@berkeley.edu}) \\
	Faculty Advisor & Steven Justice (\url{sjustice@berkeley.edu}) \\
	Meetings & Tuesdays 6--8 PM, 251 Dwinelle  \\
	Website & \url{unknowngodjournal.wordpress.com}  \\
	English 198 & 2 units 
\end{tabular}

\subsection{Course Description}
%\emph{To An Unknown God} is a journal of Christian thought at UC Berkeley.
%This class is designed to introduce students to the publishing process.
At the end of each semester, \emph{To an Unknown God}, a journal of Christian thought at UC Berkeley, publishes writing and artwork produced
by Cal students and alumni in order to exhibit the Christian intellectual tradition in an increasing secular cultural context, with the purpose of fostering dialog between students of all faiths and philosophies.
Students will criticize a broad catalog of Christian thinking from patristic sources to 21st-century developments and reflect on their interaction with selected topics, both classical (e.g.\ suffering and free) and contemporary (e.g.\ race). Students will engage these issues in writing, discourse, and discussion.
Furthermore, students will learn the fundamentals of the publishing process, viz.\ writing, editing, and design in a journal setting.

\subsection{Student Learning Objectives}
Upon the successful completion of this course, students will have experience in the following areas:
\begin{itemize}
	\item Writing thoughtful articles
	\item Editing others’ written work
	\item Designing journals
	\item Discussing contemporary topics critically and creatively in relation to Christian thought
\end{itemize}

%\subsection{Methods of Instruction}
%\begin{itemize}
%	\item Collaborative discussion
%	\item Lectures by guest speakers
%	\item Lectures by TAUG officers according to their responsibilities
%\end{itemize}
\subsection{Assignments}
Every week we will be providing a question or prompt relevant to the reading due the following week, and students will be expected to submit an approximately page-length response.

Each student will be expected to work on a longer piece throughout the semester that can be considered for the TAUG journal.

\subsection{Grading}
Your P/ NP grade for this class will be determined on a 100 point scale:

\begin{center}
	\begin{tabular}{lr}
		10 short writing projects & 50 \\
		1 long writing project & 15 \\
		Discussion participation & 25 \\
		Caf\'e Night attendance & 10
	\end{tabular}
\end{center}

\subsection{Expectations for Attendance}
TAUG meets once a week for two hours. Only two unexcused absences will be accepted; any more will result in a failing grade. Tardies (15 minutes late or more) count as one-third of an unexcused absence.

\subsection{Assigned Reading}
All materials are in a course reader for this class.
